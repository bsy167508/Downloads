\documentclass{article}
\usepackage[a4paper, total={8in, 8in}]{geometry}
\usepackage{multicol}
\setlength{\columnsep}{0.7cm}
\begin{document}
\begin{center}
\begin{Huge}
\textsf{Security Challenges Analysis of IoT}\\
\end{Huge}
Abhishek Mishra\\
Oct 11, 2016
\end{center}
\begin{multicols}{2}
\subsection*{Example of few risks}
\begin{enumerate}
\item Fitness tracker devices can be hacked to completely monitor a person's schedule.
\item Smart homes can be hacked resulting into many accidents and possible thefts.
\item Services that a person avails like car washing on service centres can be used by a hacker by accessing the data the owner carries on his IoT ID.
\item Links to a pacemaker can be fatal.
\end{enumerate}
\subsection*{Security techniques at our disposal}
\begin{enumerate}
\item Application-aware/Packet-filtering firewalls
\item Intrusion detection and prevention systems (IDS/IPS), and security incident and event management (SIEM) solutions.
\item Signature matching and blacklisting for malwares or whitelisting for more sophisticated security.
\item Virtual private networks (VPN) or physical media encryption, such as 802.11i
(WPA2) or 802.1AE (MACsec).
\end{enumerate}

\subsection*{New constraints, threats and challenges}
\begin{enumerate}
\item Applying these same practices or variants of them in the IoT world requires substantial re-engineering to address device constraints.
\item Blacklisting, for example, requires too much disk space to be practical for IoT applications.
\item Embedded devices are designed for \textbf{low
power consumption}, with a \textbf{small silicon form factor}, and often have \textbf{limited connectivity} They typically have only as much processing capacity and memory as needed for their tasks And they
are often “headless”—that is, there isn’t a human being operating them who can input authentication credentials or decide whether an application should be trusted; they \textbf{must make their own judgements and decisions} about whether to accept a command or execute a task.
\item The endless variety of IoT applications poses an equally wide variety of security challenges.

\subsection*{Building an overall secure system}
Security must be addressed throughout the device lifecycle, from the initial design to the operational environment:
\begin{enumerate}
\item \textbf{Secure booting:} When power is first introduced to the device,
the authenticity and integrity of the software on the device is
verified using cryptographically generated digital signatures
In much the same way that a person signs a check or a legal
document, a digital signature attached to the software image
and verified by the device ensures that only the software that
has been authorized to run on that device, and signed by the
entity that authorized it, will be loaded 
\item \textbf{Access control:} Next, different forms of resource and access
control are applied Mandatory or role-based access controls
built into the operating system limit the privileges of device
components and applications so they access only the resources
they need to do their jobs If any component is compromised,
access control ensures that the intruder has as minimal access
to other parts of the system as possible
\item \textbf{Device authentication:} When the device is plugged into the
network, it should authenticate itself prior to receiving or transmitting data
\item \textbf{Firewalling and IPS:} The device also needs a firewall or deep
packet inspection capability to control traffic that is destined
to terminate at the device. The device needn’t concern
itself with filtering higher-level, common Internet traffic—the
network appliances should take care of that—but it does need
to filter the specific data destined to terminate on that device
in a way that makes optimal use of the limited computational
resources available.
\item \textbf{Updates and patches:} Once the device is in operation, it will
start receiving hot patches and software updates Operators
need to roll out patches, and devices need to authenticate
them, in a way that does not consume bandwidth or impair
the functional safety of the device.
\end{enumerate}
\end{enumerate}
\end{multicols}
\end{document}